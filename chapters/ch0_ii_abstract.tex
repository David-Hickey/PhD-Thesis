\addcontentsline{toc}{chapter}{Abstract}
\begin{abstract}
    The interactions of cilia with one another and their environment are central to many important questions in biology. These hairlike organelles are found in motile and immotile (or `primary') variants, and have a variety of roles in sensing and fluid pumping. Primary cilia have long been known to act as chemosensors, but recent research has found that motile cilia also have this ability, and it is not known what benefit is conferred by combining all the complicated required molecular machinery. These chemosensitive motile cilia are often found in bundles, which is surprising, as one would expect each to deplete the local chemical concentration field, leading to a lower sensitivity per cilium. Motile cilia have long been known to synchronise with one another to produce metachronal waves, but the precise mechanism behind this synchronisation is still not well understood, except that hydrodynamics plays an important role.
    
    In this thesis, we aim to make some headway in answering these open questions, by developing models of the interactions of cilia and the surrounding fluid flow. First, using both analytical and computational methods, we determine the mass transfer to individual cilia (both primary and motile) as well as bundles of motile cilia. We show that the cilium geometry alone is sufficient to dramatically increase chemosensitivity over chemosensors on the cell surface, especially if the fluid near the cilium is in motion. We also find that motility can increase chemosensitivity by a large factor at realistic cilium speeds, and that motile bundles are more chemosensitive per cilium, provided they are beating sufficiently quickly. We then use computational methods to focus on how cilia hydrodynamically interact with one another, and show that certain cilium beats can result in strongly nonreciprocal hydrodynamic interactions that can give rise to quickly emerging metachronal order with a single dominant wavevector, even in finite systems. When the near-field hydrodynamic interactions (and hence the nonreciprocity of interactions) is suppressed, synchronisation is much slower and multiple wavevectors are seen.
    
    We have therefore uncovered several reasons why chemosensors may be advantageously located on both motile and primary cilia, and shown that a cilium beat fine-tuned to give strong nonreciprocal interactions can be extremely effective in inducing metachronal order. This amounts to a significant amount of evidence pointing to some potential answers to some of the open questions surrounding cilia.
\end{abstract}